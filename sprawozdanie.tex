%Sprawozdanie - Sprawdzian PWI

\documentclass[a4paper, 11pt]{article}
\usepackage[OT4]{fontenc}
\usepackage[utf8]{inputenc}
\usepackage[polish]{babel}
\usepackage{amsmath, amsfonts}

\begin{document}

\title{Sprawozdanie}
\author{Maksymilian Komarnicki}
\date{\today}
\maketitle

\section{Zadanie 1}

Generuje klucze poleceniem:\\
ssh-keygen -t ed25519 -C [adres emai]\\
\\
Dodaje utworzony klucz prywatny do agent-SSH poleceniem:\\
ssh-add $\tilde{\ }/.ssh/id_ed25519$\\
\\
Dodawanie klucza na serwer:\\
ssh-copy-id -i $\tilde{\ }/.ssh/id_ed25519.pub mpyzik@pwi.ii.uni.wroc.pl$\\
\\
Testuje połączenie poleceniem:\\
ssh -T mpyzik@pwi.ii.uni.wroc.pl\\
\\
Łączę się z serwerem poleceniem:\\
ssh mpyzik@pwi.ii.uni.wroc.pl\\
\\
Rozłączam się z serwerem poleceniem:\\
exit\\

\section{Zadanie 2}

Do utworzonego pliku dodaje informacje przy pomocy polecenia:\\
nano maksymilian\_komarnicki\\
\\
Doklejenie do pliku sekwencji liczb podzielnych przez 7 od 0 do 100 wykonałem za pomocą polecenia:\\
seq 0 7 100 >> maksymilian\_komarnicki.txt\\
To polecenie pozwoliło przekierowanie strumienia wyjscia pierwszej czesci polecenia do pliku
tekstowego 'maksymilian\_komarnicki.txt'. Znaki '>>' oznaczają aktualizacje pliku to znaczy sekwencja
została dopisana do końca pliku\\
\\
Plik przenoszę do katalogu testy za pomocą polecenia:\\
mv maksymilian\_komarnicki testy/\\
\\
Plik z servera na komputer lokalny przeniosłem za pomocą polecenia:\\
scp mpyzik@pwi.ii.uni.wroc.pl:/home/mpyzik/maksymilian\_komarnicki/testy/maksymilian\_komarnicki$\tilde{\ }$/sprawdzian/\\

\section{Zadanie 3}

Utworzyłem pięć plików używając polecenia:\\
hexdump /dev/urandom | head > file\\
Polecenie powtórzyłem pięciokrotnie dla plikow file1, file2, file3, file4, file5\\
\\
Wszystkie utworzone pliki skleiłem używając polecenia:\\
cat file* > concat\\
Spowodowało to przekazanie strumienia wyjściowego(zawartosci wszystkich utworzonych plikow)
do pliku concat.\\
\\
Następne zadanie zrealizowałem za pomocą polecenia:\\
grep -E '\^0.*([a-fA-F0-9][a-fA-F0-9])\\1\$' concat > output\\
Polecenie to znajduje w pliku concat odpowiednią frazę. Polecenie używa opcji -E (oznaczającą
rozszerzone wyrażenia regularne). Znajduje takie linie, które zaczynają się od znaku '0',
a kończą się powtórzoną liczbą szesnastkową, a następnie przekierowuje wyjście tego polecenia
do pliku output.\\
\\
Policzenie lini w plikach:\\
wc -l output\\
wc -l concat\\
Używając opcji -l polecenie zlicza ilość wierszy w pliku i wypisuje na wyjściu

\end{document}
